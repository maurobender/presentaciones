\documentclass[10pt]{beamer}
%\documentclass[handout, 10pt]{beamer}

\mode<handout>{
	\usepackage{pgf}
	\usepackage{pgfpages}
	
	\pgfpagesdeclarelayout{4 on 1 boxed}
{
  \edef\pgfpageoptionheight{\the\paperheight} 
  \edef\pgfpageoptionwidth{\the\paperwidth}
  \edef\pgfpageoptionborder{0pt}
}
{
  \pgfpagesphysicalpageoptions
  {%
    logical pages=4,%
    physical height=\pgfpageoptionheight,%
    physical width=\pgfpageoptionwidth%
  }
  \pgfpageslogicalpageoptions{1}
  {%
    border code=\pgfsetlinewidth{2pt}\pgfstroke,%
    border shrink=\pgfpageoptionborder,%
    resized width=.5\pgfphysicalwidth,%
    resized height=.5\pgfphysicalheight,%
    center=\pgfpoint{.25\pgfphysicalwidth}{.75\pgfphysicalheight}%
  }%
  \pgfpageslogicalpageoptions{2}
  {%
    border code=\pgfsetlinewidth{2pt}\pgfstroke,%
    border shrink=\pgfpageoptionborder,%
    resized width=.5\pgfphysicalwidth,%
    resized height=.5\pgfphysicalheight,%
    center=\pgfpoint{.75\pgfphysicalwidth}{.75\pgfphysicalheight}%
  }%
  \pgfpageslogicalpageoptions{3}
  {%
    border code=\pgfsetlinewidth{2pt}\pgfstroke,%
    border shrink=\pgfpageoptionborder,%
    resized width=.5\pgfphysicalwidth,%
    resized height=.5\pgfphysicalheight,%
    center=\pgfpoint{.25\pgfphysicalwidth}{.25\pgfphysicalheight}%
  }%
  \pgfpageslogicalpageoptions{4}
  {%
    border code=\pgfsetlinewidth{2pt}\pgfstroke,%
    border shrink=\pgfpageoptionborder,%
    resized width=.5\pgfphysicalwidth,%
    resized height=.5\pgfphysicalheight,%
    center=\pgfpoint{.75\pgfphysicalwidth}{.25\pgfphysicalheight}%
  }%
}


  \pgfpagesuselayout{4 on 1 boxed}[a4paper, border shrink=2mm, landscape]
  \nofiles
}

\usepackage[utf8]{inputenc}
\usepackage{default}
\usepackage{verbatim}
\usepackage{hyperref}

\usepackage{listings}
\usepackage{color}
\definecolor{lightgray}{rgb}{.9,.9,.9}
\definecolor{darkgray}{rgb}{.4,.4,.4}
\definecolor{purple}{rgb}{0.65, 0.12, 0.82}

\lstdefinelanguage{JavaScript}{
  keywords={typeof, new, true, false, catch, function, return, null, catch, switch, var, if, in, while, do, else, case, break},
  keywordstyle=\color{blue}\bfseries,
  ndkeywords={class, export, boolean, throw, implements, import, this},
  ndkeywordstyle=\color{darkgray}\bfseries,
  identifierstyle=\color{black},
  sensitive=false,
  comment=[l]{//},
  morecomment=[s]{/*}{*/},
  commentstyle=\color{purple}\ttfamily,
  stringstyle=\color{red}\ttfamily,
  morestring=[b]',
  morestring=[b]"
}

\lstset{
   language=JavaScript,
   backgroundcolor=\color{lightgray},
   extendedchars=true,
   basicstyle=\footnotesize\ttfamily,
   showstringspaces=false,
   showspaces=false,
   numberstyle=\footnotesize,
   numbersep=9pt,
   tabsize=3,
   breaklines=true,
   showtabs=false,
   captionpos=b
}

\usetheme{Warsaw}
\title[Javascript - JQuery (Parte 2)]{Animaciones con JQuery}
\author{Mauro Bender}
\date{Mayo 31, 2012}

\begin{document}

\begin{frame}
\titlepage
\end{frame}

\section{JQuery}
\subsection{Introducción}
\begin{frame}[fragile]
  \frametitle{JQuery}
  \begin{itemize}
    \pause \item Es una librería en javascript que nos permite, de forma fácil, modificar los elementos del DOM.
    \pause \item Para obtener los elementos del DOM usamos selectores como los de css.
    \pause \item Es facilmente extensible y debido a eso existen gran variedad de plugins..
    \pause \item Vamos a ver los como hacer animaciones con los elementos que nos proporciona esta librería.
  \end{itemize}
\end{frame}

\subsection{Cargando la librería}
\begin{frame}[fragile]
	\pause Para poder usar la librería debemos incluir el archivo javascript que la contiene. Si el archivo 
			 se encuentra en la misma carpeta que el archivo HTML, podemos usar:
	\pause \begin{lstlisting}[html]
			<script type="text/javascript" language="javascript" src="jquery.js"></script>
		\end{lstlisting}
	\pause Si no disponemos de una copia local del archivo que contiene la librería siempre podemos usar una
			 copia que se encuentra alojada en los servidores de google con el siguiente código:
	\pause \begin{lstlisting}[html]
			<script type="text/javascript" language="javascript" src="https://ajax.googleapis.com/ajax/libs/jquery/1.7.2/jquery.min.js"></script>
		\end{lstlisting}
	\pause Esto es especialmente útil cuando no se pueden subir al servidor archivos javascript.
	\pause Para que nuestro código funcione correctamente es importante que nuestro la página se cargue completamente antes
			 de que empecemos a modificarla. Para eso usamos el evento \textbf{ready} del documento:
	\pause \begin{lstlisting}[html]
			$(document).ready(function() {
				// Aca va a ir nuestro codigo...
			});
		\end{lstlisting}
\end{frame}

\section{Selectores}
\subsection{Selectores}
\begin{frame}[fragile]
	\frametitle{Selectores}
	
	\pause Para acceder a los elementos del DOM y modificarlos vamos a usar los selectores que nos proporciona
			 JQuery, como por ejemplo:
	
	\pause \begin{lstlisting}[html]
			var $links = $("a.link");
		\end{lstlisting}
		
	\pause Este selector selecciona \textbf{TODOS} los elementos \textbf{a} que tenga la clase \textbf{link}. \\
	
	\pause \$ es un alias para la clase JQuery, y por lo tanto \$() construye un nuevo objeto de la clase JQuery
			 que contiene todos los elementos a los que hace referencia el selector pasado.\\
	
	\pause Para seleccionar los elementos podemos usar expresiones de CSS.
	
	\pause Por ejemplo si tenemos un div como:
	\pause \begin{lstlisting}[html]
			<div id="box">Esto es una caja</div>
		\end{lstlisting}
		
	\pause Podemos obtenerlo con:
	\pause \begin{lstlisting}
			var elem = $("#box");
		\end{lstlisting}
	
	\pause Si en cambio queremos obtener todos los elementos con la clase ``item'', simplemente hacemos:
	
	\pause \begin{lstlisting}
			var elems = $(".item");
		\end{lstlisting}
\end{frame}

\subsection{Ejemplos}
\begin{frame}[fragile]
	\frametitle{Más ejemplos de selectores}
	
	\pause \begin{lstlisting}
			var elems = $("#lista > li");
		\end{lstlisting}
	\pause Esto selecciona todos los \emph{li}s de la lista con id \emph{lista}.
	
	\pause \begin{lstlisting}
			var elems = $("#lista li:first");
		\end{lstlisting}
	\pause Esto selecciona el primer \emph{li} de la lista con id \emph{lista}.
	
	\pause \begin{lstlisting}
			var elems = $("#lista li:last");
		\end{lstlisting}
	\pause Esto selecciona el último \emph{li} de la lista \emph{\#lista}.
	
	\pause \begin{lstlisting}
			var elems = $("div[data=username]");
		\end{lstlisting}
	\pause Esto selecciona todos los \emph{div}s que tengan el atributo \emph{data} y este sea igual a \emph{username}.
	
	\pause \begin{lstlisting}
			var elems = $("*[seleccioname]");
		\end{lstlisting}
	\pause Esto selecciona todos los elementoss que tengan el atributo \emph{seleccioname}.
	
	\pause \begin{lstlisting}
			var elems = $("a.clase-1, a.clase-2");
		\end{lstlisting}
	\pause Esto selecciona todos los \emph{a}s que tengan la clase \emph{clase-1} y o la clase \emph{clase-2}.
\end{frame}

\subsection{Selectores de JQuery}
\begin{frame}[fragile]
	\pause También disponemos de selectores que son propios de JQuery y se pueden combinar con los selectores
			 CSS que vimos anteriormente. \\
	
	\pause Por ejemplo, si queremos seleccionar el primer \emph{p} que aparece en el documento, podemos hacer:
	
	\pause \begin{lstlisting}
			var elems = $("p:eq(0)");
		\end{lstlisting}
	
	\pause Esto busca todos los \emph{p}s del documento y se queda con el primero de ellos.
	
	\pause O si queremos todos \emph{div}s que tengan la clase \emph{cajita} y estén visibles, 
			 podemos usar:
	 
	\pause \begin{lstlisting}
			var elems = $("div.cajita:visible");
		\end{lstlisting}
		
	\pause \begin{block}{Más selectores}
			Para una lista completa de todos los selectores que se pueden usar pueden ir a \url{http://docs.jquery.com/DOM/Traversing/Selectors#CSS_Selectors}.
		\end{block}
\end{frame}

\section{Modificando clases y atributos}
\subsection{Clases}
\begin{frame}[fragile]
	\frametitle{Modificando las clases}
	
	\pause Para modificar las clases que posee un grupo de elementos JQuery nos proporciona los siguientes métodos:
	
	\begin{itemize}
		\pause \item \textbf{addClass}: Nos permite agregarle una o más clases a un grupo de elementos, recibe como parámetro
				 una cadena con las clases a agregar separadas por espacios.
		\pause \begin{lstlisting}
			$("div.link").addClass("active"); // Agregamos la clase active a todos los divs con clase link
			$("p").addClass("bold big-font"); // Agregamos las clases bold y big-font a todos los ps del documento
		\end{lstlisting}
		
		\pause \item \textbf{removeClass}: Nos permite eliminar una o más clases de un elemento, recibe como parámetro las clases a eliminar.
		\pause \begin{lstlisting}
			$("div.active").removeClass("active open"); // Eliminamos las clases active y open
		\end{lstlisting}
		
		\pause \item \textbf{hasClass}: Nos permite saber si un elemento tiene o no una clase. Recibe como parámetro la clase
			que se quiere verificar y devuelve \textbf{true} en caso que la contenga y \textbf{false} en caso contrario.
		\pause \begin{lstlisting}
		$("div.link").hasClass("active"); // true
		\end{lstlisting}
	\end{itemize}
\end{frame}

\subsection{Atributos}
\begin{frame}[fragile]
	\frametitle{Modificando los atributos}
	
	\pause Para modificar las atributos de los elementos disponemos de los siguientes métodos:
	
	\begin{itemize}
		\pause \item \textbf{attr}: Nos permite obtener el valor de un atributo del primer elemento que coincida con el selector.
			Debemos pasarle como parámetro el nombre del atributo. Nos devuelve el valor del atributo si éste está seteado o
			undefined en caso de no estarlo.
			\pause \begin{lstlisting}
				var url = $("a.link").attr("href");
			\end{lstlisting}
			
			\pause \textbf{attr} también nos permite setear los atributos de uno o más elementos. Para esto disponemos de dos formas:
			\begin{itemize}
				\pause \item Si queremos setear sólo un atributo, le podemos pasar como parámetros el nombre del atributo y el valor.
				\pause \begin{lstlisting}
					$("img").attr("src", "http://server.com/imagen.jpg");
				\end{lstlisting}
				
				\pause \item Si en cambio queremos setear varios atributos al mismo tiempo le podemos pasar un objeto.
				\pause \begin{lstlisting}
					$("img#foto").attr({"alt" : "Una foto", "title" : "con title"});
				\end{lstlisting}
			\end{itemize}
		\end{itemize}
\end{frame}

\begin{frame}[fragile]
	\frametitle{Modificando los atributos - continuación}
	\begin{itemize}
		\pause \item \textbf{removeAttr}: Nos permite eliminar una o más clases de un elemento, recibe como parámetro las clases a eliminar.
		\pause \begin{lstlisting}
			$("div.active").removeClass("active open"); // Eliminamos las clases active y open
		\end{lstlisting}
	\end{itemize}
\end{frame}

\section{Modificando el css}
\subsection{Modificando el css}
\begin{frame}[fragile]
	\frametitle{Modificando el css}
	
	\pause Para modificar los estilos de un elemento siemplemente usamos el método \textbf{css} del objeto que
		hace referencia al elemento a modificar. \\
		
	\pause Hay dos formas de llamar al método css, una es pasandole dos argumentos: la propiedad a modificar
		y el valor nuevo de dicha propiedad:
	\pause \begin{lstlisting}
			$("selector").css("propiedad", "valor");
		\end{lstlisting}
		
	\pause O, si queremos modificar varias propiedades a la vez, un objeto con las propiedades a modificar como
		claves y los nuevos valores de esas propiedades como los valores de dichas claves:
	\pause \begin{lstlisting}
			$("#box").css({
			   "height" : "300px",
			   "width" : "400px"
			});
		\end{lstlisting}
	
	\pause Algo que hay que tener en cuenta es que las propiedades que en css tienen nombres compuestos, separados
		por ``-'' en JQuery deben ir todas las palabras juntas, borrando los ``-'', y cada nueva palabra debe comenzar
		con una mayúscula. Por ejemplo: ``margin-left'' pasa a ser ``marginLeft''.
\end{frame}

\begin{frame}[fragile]
	\pause El método css tambiés nos devuelve el valor de una propiedad del estilo si le pasamos como parámetro una cadena
		con el nombre de dicha propiedad. Por ejemplo, si queremos saber si un elemento está siendo mostrado en la página
		podríamos hacer:
	\pause \begin{lstlisting}
			var display = $("#box").css("display");
		\end{lstlisting}
	\pause \begin{lstlisting}
			if(display == "none") {
			   alert("Está oculto");
			} else {
			   alert("Es visible");
			}
		\end{lstlisting}
	
	\pause Ahora podríamos hacer que el elemento se muestre o se oculte dapendiendo de si está siendo mostrado o no.
	
	\pause Para mostrar un elemento podríamos usar:
	\pause \begin{lstlisting}
			$("#box").css("display", "block");
		\end{lstlisting}
	\pause Esto funcionaría, pero JQuery nos proporciona un método que hace esto de forma automática. Este método es
		\textbf{show} y se usa de la siguiente forma:
	\pause \begin{lstlisting}
			$("#box").show();
		\end{lstlisting}
\end{frame}

\begin{frame}[fragile]
	\pause Análogamente disponemos del método \textbf{hide}, que oculta un elemento del DOM:
	\pause \begin{lstlisting}
			$("#box").hide();
		\end{lstlisting}
	\pause Esto sería lo mismo que hacer:
	\pause \begin{lstlisting}
			$("#box").css("display", "none");
		\end{lstlisting}
	
	\pause Ahora uniendo todo, podríamos crear una función toggle a la que le pasamos una cadena con el
		selector de un elemento y oculta dicho elemento si este está siendo mostrado o lo muestra si no.
		Esta función quedaría como (en el siguiente slide): 
\end{frame}

\begin{frame}[fragile]
 \pause \begin{lstlisting}
			function cambiar(selector) {
			   var elem = $(selector);
			   
			   if(elem.css("display") == "none") {
			      elem.show(); // Mostramos el elemento
			   } else {
			      elem.hide(); // Ocultamos el elemento
			   }
			}
			
			cambiar("#box");
		\end{lstlisting}
	\pause Esta función que hicimos ya existe en JQuery y se llama toggle, se usa de la siguiente forma:
	\pause \begin{lstlisting}
			$("#box").toggle();
		\end{lstlisting}
\end{frame}


\section{Eventos}
\subsection{Funciones para manejar eventos}
\begin{frame}[fragile]
	\frametitle{Eventos}
	\pause Los eventos son ``avisos'' que son lanzados cuando ocurre algún tipo de cambio sobre los elementos del DOM.
	
	\pause Existen varios tipos de eventos. Algunos responden a acciones que realiza el usuarios,
			como: ``click'', ``hover'', ``focus'', ``submit'', etc; mientras que otros son consecuencia de cambios en
			la página, como: ``ready'', ``load''.
	
	\pause Muchas veces nos interesa realizar una acción cuando cierto evento ocurre. Para esto JQuery nos
			proporciona las siguientes funciones:
	
	\begin{itemize}
		\pause \item \textbf{bind}: recibe dos parámentros, una cadena con el evento que se quiere ``escuchar'' y una función
				que se va a ejecutar cada vez que el evento es lanzado. Por ejemplo si queremos mostrar un alert cada vez
				que se hace click en algún \emph{a} de la página podemo hacer:
				\begin{lstlisting}
					$("a").bind("click", function(event) {alert("Se ha presionado un a");});
				\end{lstlisting}
	\end{itemize}
\end{frame}

\begin{frame}[fragile]
	\begin{itemize}
		\pause \item \textbf{unbind}: Produce que una función que se enlazó a un evento usando \textbf{bind} deje de ejecutarse
				cada vez que se lanza dicho evento. Esto sólo funcionara si la función que se enlazó al evento no era una
				función anonima. 
				\pause \begin{lstlisting}
					function alertarClick(event) {alert("Se ha presionado un link");}
					
					$("a").bind("click", alertarClick);
					$("a").unbind("click", alertarClick);
				\end{lstlisting}
				
				\pause Si en cambio se quiere eliminar todas las funciones que están enlazadas a cualquier evento
					de un grupo de elementos usamos:
				\pause \begin{lstlisting}
					$("a").unbind();
				\end{lstlisting}
				
				\pause O si se quiere eliminar todas las funciones enlazadas a un evento especifico de un grupo de
					elementos podemos usar:
				\pause \begin{lstlisting}
					$("a").unbind("click");
				\end{lstlisting}
	\end{itemize}
\end{frame}

\subsection{Atajos para enlazar eventos}
\begin{frame}[fragile]
	\pause Existen varios atajos para bindear eventos, algunos son:
	\begin{itemize}
		\pause \item \textbf{click}:
		\pause \begin{lstlisting}
					$("a").bind("click", function(event){});
					$("a").click(function(event){});
				\end{lstlisting}
		\pause \item \textbf{submit}:
		\pause \begin{lstlisting}
					$("form").bind("submit", function(event){});
					$("form").submit(function(event){});
				\end{lstlisting}
		\pause \item \textbf{change}:
		\pause \begin{lstlisting}
					$("select[name=pais]").bind("change", function(event){});
					$("select[name=pais]").change(function(event){});
				\end{lstlisting}
		\pause \item \textbf{ready}:
		\pause \begin{lstlisting}
					$(document).bind("ready", function(event){});
					$(document).ready(function(event){});
				\end{lstlisting}
	\end{itemize}
\end{frame}

\begin{frame}[fragile]
	\begin{itemize}
		\pause \item \textbf{hover}: es un caso especial en donde se le pueden pasar dos funciones
			como parámetros en vez de sólo una. La primera función se va a ejecutar cuando el puntero
			del mouse entre en el elemento y la segunda cuando éste salga.
		\pause \begin{lstlisting}
					$("a").bind("mouseenter", function(e){alert("entro");});
					$("a").bind("mouseleave", function(e){alert("salio");});
					
					$("a").hover(
					   function(e){alert("entro");},
					   function(e){alert("salio");}
					);
				\end{lstlisting}
	\end{itemize}
	
	\pause \begin{block}{El objeto Event}
		Las funciones que enlazamos a los eventos reciben como primer parámetro un objeto que representa al evento
		lanzado. Este objeto posee, entre otras cosas, una referencia al elemento que lanzó el objeto y algunas funciones
		que nos pueden ser de utilidad. Dentro de ellas, \emph{preventDefault} nos permite evitar que el objeto realice
		la acción que tiene asociada para ese evento por defecto.
	\end{block}
\end{frame}

\begin{frame}
	\begin{center}
		Fin =).
	\end{center}
\end{frame}


\end{document}
 para esta librería