%\documentclass[10pt]{beamer}
\documentclass[handout, 10pt]{beamer}

\mode<handout>{
	\usepackage{pgf}
	\usepackage{pgfpages}
	
	\pgfpagesdeclarelayout{4 on 1 boxed}
{
  \edef\pgfpageoptionheight{\the\paperheight} 
  \edef\pgfpageoptionwidth{\the\paperwidth}
  \edef\pgfpageoptionborder{0pt}
}
{
  \pgfpagesphysicalpageoptions
  {%
    logical pages=4,%
    physical height=\pgfpageoptionheight,%
    physical width=\pgfpageoptionwidth%
  }
  \pgfpageslogicalpageoptions{1}
  {%
    border code=\pgfsetlinewidth{2pt}\pgfstroke,%
    border shrink=\pgfpageoptionborder,%
    resized width=.5\pgfphysicalwidth,%
    resized height=.5\pgfphysicalheight,%
    center=\pgfpoint{.25\pgfphysicalwidth}{.75\pgfphysicalheight}%
  }%
  \pgfpageslogicalpageoptions{2}
  {%
    border code=\pgfsetlinewidth{2pt}\pgfstroke,%
    border shrink=\pgfpageoptionborder,%
    resized width=.5\pgfphysicalwidth,%
    resized height=.5\pgfphysicalheight,%
    center=\pgfpoint{.75\pgfphysicalwidth}{.75\pgfphysicalheight}%
  }%
  \pgfpageslogicalpageoptions{3}
  {%
    border code=\pgfsetlinewidth{2pt}\pgfstroke,%
    border shrink=\pgfpageoptionborder,%
    resized width=.5\pgfphysicalwidth,%
    resized height=.5\pgfphysicalheight,%
    center=\pgfpoint{.25\pgfphysicalwidth}{.25\pgfphysicalheight}%
  }%
  \pgfpageslogicalpageoptions{4}
  {%
    border code=\pgfsetlinewidth{2pt}\pgfstroke,%
    border shrink=\pgfpageoptionborder,%
    resized width=.5\pgfphysicalwidth,%
    resized height=.5\pgfphysicalheight,%
    center=\pgfpoint{.75\pgfphysicalwidth}{.25\pgfphysicalheight}%
  }%
}


  \pgfpagesuselayout{4 on 1 boxed}[a4paper, border shrink=2mm, landscape]
  \nofiles
}

\usepackage[utf8]{inputenc}
\usepackage{default}
\usepackage{verbatim}
\usepackage{hyperref}

\usepackage{listings}
\usepackage{color}
\definecolor{lightgray}{rgb}{.9,.9,.9}
\definecolor{darkgray}{rgb}{.4,.4,.4}
\definecolor{purple}{rgb}{0.65, 0.12, 0.82}

\lstdefinelanguage{JavaScript}{
  keywords={typeof, new, true, false, catch, function, return, null, catch, switch, var, if, in, while, do, else, case, break},
  keywordstyle=\color{blue}\bfseries,
  ndkeywords={class, export, boolean, throw, implements, import, this},
  ndkeywordstyle=\color{darkgray}\bfseries,
  identifierstyle=\color{black},
  sensitive=false,
  comment=[l]{//},
  morecomment=[s]{/*}{*/},
  commentstyle=\color{purple}\ttfamily,
  stringstyle=\color{red}\ttfamily,
  morestring=[b]',
  morestring=[b]"
}

\lstset{
   language=JavaScript,
   backgroundcolor=\color{lightgray},
   extendedchars=true,
   basicstyle=\footnotesize\ttfamily,
   showstringspaces=false,
   showspaces=false,
   numberstyle=\footnotesize,
   numbersep=9pt,
   tabsize=3,
   breaklines=true,
   showtabs=false,
   captionpos=b
}

\usetheme{Warsaw}
\title[Javascript - JQuery (Parte 1)]{Introducción a JavaScript}
\author{Mauro Bender}
\date{Mayo 28, 2012}

\begin{document}

\begin{frame}
\titlepage
\end{frame}

\section{Javascript}
\subsection{Introducción}
\begin{frame}
  \frametitle{Javascript}


  \begin{itemize}
    \pause \item Es un lenguaje de programación que se ejecuta del lado del cliente, es decir, se ejecuta en el navegador del usuario.
    \pause \item Nos permite acceder al DOM (Document Object Model) del documento de forma ``fácil''.
    \pause \item Nos proporciona herramientas para poder manipular el DOM.
    \pause \item Vamos a ver los aspectos básicos del lenguaje y herramientas que nos van a ser útiles al momento de hacer nuestras animaciones.
  \end{itemize}
  
  \onslide<3->
  \begin{block}{DOM}
	El DOM (Document Object Model) es un interfaz a través de la cual nos es posible acceder y modificar el contenido,
	estructura y estilo de los documentos HTML (es lo que nos permite acceder a todas esas cositas bonitas que queremos
	cambiar y animar). \\
	
	Mas información: \url{http://es.wikipedia.org/wiki/Document_Object_Model}.
  \end{block}

\end{frame}

\section{Variables}
\subsection{Declarar variables}
\begin{frame}[fragile]
	\frametitle{Variables}
	En javascript las variables no son más que ``contenedores'' que nos permiten almacenar distintos tipos de valores.
	
	\pause
	
	Para declarar una variable usamos la palabra reservada \textbf{var} como sigue:
	
	\begin{lstlisting}
		var nombre_variable;
	\end{lstlisting}
	
	\pause
	
	Los nombres de variables pueden contener caracteres alfanuméricos y los símbolos $\$$ y $\_$, con la salvedad
	de que no pueden comenzar con un número.
	
	\pause
	
	Las variables se pueden inicializar al declararlas o en cualquier momento posterior en el programa, para esto
	usamos el operador de $=$:
	\begin{lstlisting}
		var numero = 5;
		var nombre;
		nombre = "Margarito Flores";
	\end{lstlisting}
	
\end{frame}

\subsection{Tipos de variables}
\begin{frame}[fragile]
	\frametitle{Tipos de variables}
	En javascript las variables pueden ser de diferentes tipos: 
	\begin{itemize}
		\pause \item \textbf{Numeric}: Variables numéricas como: 1, 4, 4.
		\pause \item \textbf{String}: Cadenas de caracteres (texto) como: ``Hola mundo!'', ``Dieguín!!''.
		\pause \item \textbf{Boolean}: Valores de verdad: \textbf{true} o \textbf{false}.
		\pause \item \textbf{Array}: Es una lista de elementos indexada por un número positivo:
			\begin{lstlisting}
				var colores = ["rojo", "azul", "verde"];
			\end{lstlisting}
			
		\pause \item \textbf{Object}: Objetos, pares de clave valor como por ejemplo:
			\begin{lstlisting}
				var objeto = {"nombre": "mauro", "edad" : 24};
			\end{lstlisting}
	\end{itemize}
	
	No es necesario especificar el tipo de valor que va a contener la variable, javascript lo determina
	al momento en que se le asigna un valor.
\end{frame}

\subsection{Los tipos String y Array}
\begin{frame}[fragile]
	\frametitle{Tipo String y Array}
	\pause Las variables de tipo String y Array proporcionan un conjunto de propiedades y métodos que son
			útiles cuando se esta trabajando con ellas. \\
	
	\pause De ellas una de las más importantes en ``length'' que nos devuelve el tamaño de la cadena
	o del arreglo.
	
	\pause
	\begin{lstlisting}
		var cadena  = "Esto es una cadena ;-)";
		var arreglo = ["esto", "es", "un", "arreglo", ":P"];
		
		cadena.length;  // 22
		arreglo.length; // 5
	\end{lstlisting}
	
	\pause
	\begin{block}{Más información}
		Para saber que otras propiedades y/o métodos poseen los Arrays y los Strings pueden ir a 
		\url{http://www.w3schools.com/jsref/jsref_obj_string.asp} (para información sobre el tipo
		String) o a \url{http://www.w3schools.com/jsref/jsref_obj_array.asp} (para información sobre
		el tipo Array).
	\end{block}
\end{frame}

\subsection{Comentarios}
\begin{frame}[fragile]
	\begin{block}{Comentarios}
		Los comentarios son porciones de textos que podemos a agregar a nuestro código para hacerlo más
		entendible. Éstos no afectan la ejecución del código ya que son ignorados por el navegador. Para
		escribir un comentario podemos usar \textbf{//} o \textbf{/*} \ldots \textbf{*/}:
		\begin{lstlisting}
			// Comentario de una linea
			
			/* Comentario
			 de
			 varias
			 lineas
			*/
		\end{lstlisting}
  \end{block}
\end{frame}


\section{Operadores}
\subsection{Operadores aritméticos}
\begin{frame}[fragile]
	\frametitle{Operadores - Operadores aritméticos}
	Existen diferentes tipos operadores en JavaScript
	\begin{itemize}
		\pause \item \textbf{+}: Efectúa la suma de dos números.
			\begin{lstlisting}
				var suma = 5 + 4; // suma ahora vale 9
			\end{lstlisting}
		
		\pause \item \textbf{-}: Se utiliza para restar números.
			\begin{lstlisting}
				var resta = 5 - 4; // resta ahora vale 1
			\end{lstlisting}
		
		\pause \item \textbf{*}: Sirve para multiplicar dos números
			\begin{lstlisting}
				var resultado = 2 * 3; // resultado ahora vale 6
			\end{lstlisting}
		
		\pause \item \textbf{/}: Se utiliza para dividir dos números.
			\begin{lstlisting}
				var resultado = 6 / 3; // resultado ahora vale 2
			\end{lstlisting}
		
		\pause \item \textbf{\%}: Se utiliza para obtener el modulo de una división (resto de la división entera).
			\begin{lstlisting}
				var resultado = 7 % 3; // resultado ahora vale 1
			\end{lstlisting}
	\end{itemize}
\end{frame}

\subsection{Operadores aritméticos unarios}
\begin{frame}[fragile]
	\frametitle{Operadores - Operadores aritméticos unarios}
	\begin{itemize}
		\pause \item \textbf{++}: Incrementa en uno la variable afectada.
			\begin{lstlisting}
				var numero = 1;
				numero++; // numero ahora vale 2
			\end{lstlisting}
			\pause Es lo mismo que ejecutar:
			\begin{lstlisting}
				var numero = 1;
				numero = numero + 1; // numero ahora vale 2
			\end{lstlisting}
		\pause \item \textbf{$--$}: Análogo a \textbf{++}. Decrementa en uno la variable afectada.
			\begin{lstlisting}
				var numero = 1;
				numero--; // numero ahora vale 0
			\end{lstlisting}
			\pause Es lo mismo que ejecutar:
			\begin{lstlisting}
				var numero = 1;
				numero = numero - 1; // numero ahora vale 0
			\end{lstlisting}
	\end{itemize}
\end{frame}


\subsection{Operadores de comparación}
\begin{frame}[fragile]
	\frametitle{Operadores - Operadores de comparación}
	\begin{itemize}
		\pause \item \textbf{==}: Devuelve \textbf{true} si los dos operandos son iguales. \textbf{false} en cualquier otro caso.
			\begin{lstlisting}
				var num1 = 5, num2 = 3, num3 = 5;
				num1 == num2; // false
				num1 == num3; // true
			\end{lstlisting}
			
		\pause \item \textbf{!=}: Devuelve \textbf{true} si los dos operandos son distintos. \textbf{false} en cualquier otro caso.
			\begin{lstlisting}
				var nombre1 = "javier", nombre2 = "adri", nombre3 = "javier";
				nombre1 != nombre2; // false
				nombre1 != nombre3; // true
			\end{lstlisting}
		
		\pause \item \textbf{!}: Niega la expresión afectada. Devuelve \textbf{true} si expresión es falsa y \textbf{false} si es verdadera.
			\begin{lstlisting}
				var falso     = !true; // falso es igual a false
				
				var distintos =  !("iguales" == "iguales"); // Es lo mismo que hacer "iguales" != "iguales"
			\end{lstlisting}
	\end{itemize}
\end{frame}

\begin{frame}[fragile]
	\begin{itemize}
		\pause \item \textbf{===}: Compara por igualdad pero con un criterio más fuerte que ==. Devuelve \textbf{true} si los dos
						  operandos contienen el mismo valor y además son del mismo tipo. \textbf{false} en cualquier otro caso.
			\begin{lstlisting}
				var num1 = 5, // Es del tipo Numeric
				    num2 = "5", // Es del tipo String
				    num3 = 5; // Es del tipo Numeric
				num1 === num2; // false
				num1 === num3; // true
			\end{lstlisting}
			
		\pause \item \textbf{!==}: Análogo a \textbf{===}. Devuelve \textbf{true} si los operandos son de distinto tipo o si si contienen
		             distintos valores. \textbf{false} en cualquier otro caso.
			\begin{lstlisting}
				var num1 = 5, // Es del tipo Numeric
				    num2 = "5", // Es del tipo String
				    num3 = 3; // Es del tipo Numeric
				num1 !== num2; // true
				num1 !== num3; // true
			\end{lstlisting}
	\end{itemize}
\end{frame}

\begin{frame}[fragile]
	\begin{itemize}
		\pause \item $\boldmath{<, <=, >, >=}$: Devuelven \textbf{true} si el primer operando es menor, menor o igual, mayor o
		mayor o igual que el segundo operando respectivamente. \textbf{false} en otro caso.
			\begin{lstlisting}
				var num1 = 5, num2 = 3, num3 = 5;
				num1 < num2;  // false
				num1 <= num3; // true;
				num1 > num3;  // false
				num1 >= num2; // true
			\end{lstlisting}
	\end{itemize}
	
	\pause
	
	\begin{block}{Para tener en cuenta}
      El operador $+$ también sirve para concatenar cadenas y arrays.
      \begin{lstlisting}
			var nombre = "Mauro" + " " + "Bender";
				// nombre ahora es "Mauro Bender"
			
			var colores = ["azul", "rojo"] + ["blanco", "negro"];
				// colores ahora es ["azul", "rojo", "blanco", "negro"]
		\end{lstlisting}
   \end{block}
\end{frame}

\section{Funciones}
\subsection{Declarando funciones}
\begin{frame}[fragile]
	\frametitle{Funciones}
	Las funciones son bloques de códigos que reciben parámetros y pueden devolver un valor. Para declarar una función usamos
	la palabra reservada \textbf{function} y encerramos el código de la función entre \{ \}.
	
	\begin{lstlisting}
		function nombre_de_la_funcion (param1, param2, ...) {
		   // codigo de la funcion
		}
	\end{lstlisting}
	
	\pause
	
	La palabra reservada \textbf{return} se utiliza para indicar que se debe terminar la ejecución de la función 
	y devolver el valor que se indica a continuación. Si no se indica, la función termina al encontrar la \} que
	cierra el bloque y no devuelve ningún valor.
	
	\begin{lstlisting}
		function suma(numero1, numero2) {
		   return numero1 + numero2;
		}
	\end{lstlisting}
	
	\pause
	
	Para llamar a una función sólo debemos escribir su nombre pasándole entre paréntesis los valores que queremos que
	reciba.
	
	\begin{lstlisting}
		var res = suma(3, 6); // res es ahora 9
	\end{lstlisting}
\end{frame}

\subsection{Funciones como valores de variables}
\begin{frame}[fragile]
	En javascript las funciones son interpretadas como valores, por lo que las podemos asignar a una variable.
	
	\begin{lstlisting}
		var suma = function(numero1, numero2) {return numero1 + numero2;}
		var res = suma(3, 6);
	\end{lstlisting}
	
	\pause
	
	Por lo tanto, también es posible pasarlas como un parámetro a otra función.
	
	\begin{lstlisting}
		function alertarResultado (funcion, param1, param2) {
			var res = funcion(param1, param2);
		   alert(res);
		}
		function suma(num1, num2) {return num1 + num2;}
		
		alertarResultado(suma, 3, 6);
	\end{lstlisting}
\end{frame}

\section{Estructuras de control}
\subsection{If... Else...}
\begin{frame}[fragile]
	\frametitle{Estructuras de control}
	En javascript disponemos también de ciertas estructuras que nos permiten controlar el flujo de
	ejecución de un programa. \\ 
	\vspace*{10pt}
	
	\textbf{\large{If... Else...}} \\
	
	Con la estructura \textbf{if... else...} podemos decirle al programa que ejecute una porción de código
	o no en base a una condición.
	
	\begin{lstlisting}
		if(condicion) {
			// hace algo si es condicion es verdadera
		} else {
			// hacer otra cosa en otra caso
		}
	\end{lstlisting}
	
	\pause
	
	El bloque correspondiente al \textbf{else} que sólo se ejecuta cuando la condición es false es
	opcional y se puede omitir si uno así lo desea.
\end{frame}

\begin{frame}[fragile]
	\pause
	
	En el siguiente ejemplo mostramos una alerta indicando si un número ingresado por el usuario es igual
	o no a 3, usando un if.
	
	\begin{lstlisting}
		var numero = prompt("Ingrese un numero:");
		
		if(numero == 3) {
			alert("Es igual a 3 =).");
		} else {
			alert("No es igual a 3 =(.");
		}
	\end{lstlisting}
	
	\pause
	
	\begin{block}{prompt y alert}
		\textbf{prompt} y \textbf{alert} son dos ``popups'' que nos proporciona javascript. \textbf{prompt} sirve para pedirle al usuario
		que ingrese un valor mientras que \textbf{alert} se usa comúnmente para mostrarle información al usuario. \\
		
		Mas información: \url{http://www.w3schools.com/js/js_popup.asp}.
	\end{block}

\end{frame}

\subsection{While}
\begin{frame}[fragile]
	\textbf{\large{While}} \\
	
	\textbf{while} sirve para repetir una porción de código mientras una condición siga siendo válida. Su sintaxis
	es:
	
	\begin{lstlisting}
		while(condicion) {
			// hace algo mientras condicion sea verdadera
		} 
	\end{lstlisting}
	
	\pause
	
	Por ejemplo podemos tener un programa que vaya decrementando un número ingresado por el usuario en 1 hasta llegar
	a 0, alertando en cada paso el número que va quedando.
	
	\begin{lstlisting}
		var numero = prompt("Ingrese un numero mayor a 0: ");
		
		while(numero > 0) {
			alert("Numero ahora todavia no es 0: " + numero);
			
			numero--;
		} 
		
		alert("Llegamos a cero");
	\end{lstlisting}
\end{frame}

\subsection{For}
\begin{frame}[fragile]
	\textbf{\large{For}} \\
	
	\textbf{for} es también una estructura de repetición como \textbf{while}, pero es usualmente usado para ejecutar
	una porción de códigos una cantidad fija de veces.
	
	\begin{lstlisting}
		for(incializacion; condicion; incremento) {
			// codigo
		} 
	\end{lstlisting}
	
	\pause
	
	Como se puede ver la estructura \textbf{for} se compone de cuatro partes: 
	\begin{itemize}
		\pause \item \textbf{inicializacion}: Aquí debemos, como su nombre lo indica, inicializar las variables que vamos
		      a usar para controlar el loop.
		\pause \item \textbf{condicion}: Acá debemos escribir una condición que se tiene que cumplir para que el bucle se
				siga ejecutando.
		\pause \item \textbf{incremento}: Una porción de código que se va a ejecutar cada vez que se haya completado un
				ciclo del bucle, usualmente usado para incrementar o decrementar la variable que se usa para controlar
				el bucle.
		\pause \item \textbf{codigo}: El código que se va a ejecutar en cada repetición.
	\end{itemize}
\end{frame}

\begin{frame}[fragile]
	\pause
	
	Ejemplo:
	
	\begin{lstlisting}
		var colores = ["rojo", "verde", "azul", "blanco"];
		
		for(var i = 0; i < colores.length; i++) {
			alert("El color actual es: " + colores[i]);
		}
	\end{lstlisting}
	
	\pause En el ejemplo creamos un array con diferentes colores y después lo recorremos y mostramos
	       los colores que contiene usando un bucle \textbf{for}.

\end{frame}

\section{Repaso}
\subsection{¿Qué vimos?}
\begin{frame}[fragile]
	\frametitle{Repaso}
	
	\pause ¿Qué vimos?
	
	\begin{itemize}
		\pause \item Variables
			\begin{itemize}
				\pause \item Como declarar variables.
				\pause \item Diferentes tipos de variables disponibles en Javascript.
			\end{itemize}
		
		\pause \item Operadores
			\begin{itemize}
				\pause \item Operadores aritméticos.
				\pause \item Operadores de comparación.
				\pause \item Operadores cuya significado depende del tipo de variables a la que se aplica (``+'').
			\end{itemize}
		
		\pause \item Funciones
			\begin{itemize}
				\pause \item Diferentes formas de declarar funciones.
				\pause \item Funciones que reciben funciones como parámetros.
			\end{itemize}
		
		\pause \item Estructuras de control de flujo
			\begin{itemize}
				\pause \item If... Else...
				\pause \item While(){}
				\pause \item For(;;){}
			\end{itemize}
	\end{itemize}
\end{frame}

\subsection{¿Qué vamos a ver?}
\begin{frame}[fragile]
	\pause ¿Qué vamos a ver el jueves?
	
	\begin{itemize}
		\pause \item La librería JQuery: ¿Qué es? ¿Cómo se usa?
		\pause \item Uso de los selectores para obtener los elementos de la página.
		\pause \item Cómo modificar el css de un elemento desde javascript.
		\pause \item Animaciones incorporadas en JQuery.
		\pause \item Crear nuestras animaciones con el método ``animate''.
	\end{itemize}
\end{frame}

\begin{frame}
	\begin{center}
		Fin =).
	\end{center}
\end{frame}


\end{document}
